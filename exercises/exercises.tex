\documentclass{article}
\usepackage[utf8]{inputenc}
\usepackage{newunicodechar}
\usepackage{stickstootext}
\usepackage[stix2,vvarbb]{newtxmath}
\usepackage{bussproofs}

\newunicodechar{⊆}{\subseteq}
\newunicodechar{∊}{\in}
\newunicodechar{∪}{\cup}
\newunicodechar{⇒}{\Rightarrow}
\newunicodechar{×}{\times}
\newunicodechar{≤}{\leq}

\newcommand{\tf}[1]{\texttt{#1}}
\newcommand{\tc}[1]{\mathcal{#1}}
\newcommand{\R}{\mathbin R}
\newcommand{\true}{\texttt{true}}
\newcommand{\false}{\texttt{false}}
\newcommand{\iszero}{\mathop\texttt{iszero}}
\newcommand{\xsucc}{\mathop\texttt{succ}}
\newcommand{\pred}{\mathop\texttt{pred}}
\newcommand{\xif}[3]{\texttt{if}\,#1\texttt{then}\,#2\,\texttt{else}\,#3}

\begin{document}
\begin{itemize}
\item[2.2.6] It is clear that $R'$ is reflexive. Suppose $R''$ is
some other reflexive relation containing $R$, which is to say
$(s, s) ∊ R''$ for all $s ∊ S$ and $R ⊆ R''$. Hence
\begin{equation*}
R' = R ∪ \{(s, s) \mid s ∊ S\} ⊆ R'',
\end{equation*}
so $R'$ is smallest.

\item[2.2.7] First note that $R$ is cumulative, i.e.
\begin{equation}\label{eq:227}
n ≤ m ⇒ R_n ⊆ R_m.
\end{equation}
(We leave it as an unproven fact that for any sequence $s$, the
two propositions `$s_n ≤ s_{n + 1}$ for all $n$' and `$s_n ≤ s_m$
for all $n ≤ m$' are equivalent.)

Suppose $s \R^+ t$ and $t \R^+ u$. Since $R^+$ is a union,
there must be $m, n$ such that $s \R_m t$ and $t \R_n u$.
From (\ref{eq:227}) it follows that $s \R_{\max \{m, n\}} t$ and
$t \R_{\max \{m, n\}} u$, so by construction
$s \R_{\max \{m, n\} + 1} u$, which implies $s \R^+ u$. Hence
$R^+$ is transitive.

Suppose $R''$ is some other transitive relation containing $R$.
Thus $R_0 = R ⊆ R''$, and $R_n ⊆ R'' ⇒ R_{n+1} ⊆ R''$ since $R''$
is transitive, so $R_n ⊆ R''$ for all $n$. Hence $R^+ ⊆ R''$, so
$R^+$ is smallest.

\item[2.2.8] Suppose $P$ is preserved by $R$. Since
$P\,s ⇒ P\,s$, adding $(s, s)$ to $R$ can never invalidate the
preservation of $P$ by $R$. In a similar fashion, if
$P\,s$, $s \R t$, and $t \R u$, then $P\,t$ and so $P\,u$ by the
preservation of $P$ by $R$, so adding $(s, u)$ to $R$ can do no
harm. In conclusion, $P$ is preserved by $R^*$.

\item[3.2.4] Let $n_i$ be the size of $S_i$. Then $n_0 = 0$ and
$n_{i + 1} = 3 + 3 n_i + n_i^3$, so $n_3 = 59~439$.

\item[3.2.5] Let $F$ be the construction such that $S_{i + 1} =
F(S_i)$. We show that $F$ is monotone w.r.t.\ inclusion, i.e.
\begin{equation}\label{eq:325}
A ⊆ B ⇒ F(A) ⊆ F(B).
\end{equation}
Suppose $A ⊆ B$ and $t ∊ F(A)$. Then either
\begin{itemize}
\item $\tf t ∊ \{\true, \false, \tf 0\}$,
in which case $\tf t ∊ F(B)$, or
\item $\tf t ∊ \{\xsucc \tf t_1, \pred \tf t_1,
\iszero \tf t_1\}$ for some $t_1 ∊ A$,
in which case $t_1 ∊ B$, so $t ∊ F(B)$, or
\item $\tf t = \xif{\tf t_1}{\tf t_2}{\tf t_3}$ for
$\{\tf t_1, \tf t_2, \tf t_3\} ⊆ A$,
in which case $\{t_1, t_2, t_3\} ⊆ B$, so $t ∊ F(B)$.
\end{itemize}
Hence $F(A) ⊆ F(B)$.

We can now see that $S$ is cumulative, since $S_0 ⊆ S_1$ and also
$S_i ⊆ S_{i + 1} ⇒ S_{i + 1} ⊆ S_{i + 2}$, where the first fact
follows from $S_0$ being the empty set, and the second fact is a
substitution instance of (\ref{eq:325}).

\item[3.3.4] Let $\bar P$ be the extension of the property $P$,
i.e.\ $\bar P = \{\tf s \in \tc T \mid P(\tf s)\} ⊆ \tc T$. The
premises of structural induction assert that $\bar P$ satisifies
conditions 1--3 in definition 3.2.1, and since $\tc T$ is the
smallest such set, $\tc T ⊆ \bar P$. Thus $\bar P = \tc T$ and so
$P(\tf s)$ for all $\tf s ∊ \tc T$, which was to be shown.

Note that, by definition, all terms in $\tc T$ have a strictly
larger both depth and size than any of its (immediate) subterms,
so the premises of induction by on depth and size imply the
premises for structural induction, and the conclusion follows.

(Another way to validate induction on depth would be to prove
that $\{\tf s ∊ \tc T \mid \textsl{depth}(s) < i\} = S_i$, and go
from there.)

\item[3.5.5] Structural induction.

\item[3.5.10]
\begin{equation*}
\AxiomC{$\tf t \longrightarrow \tf t'$}
\UnaryInfC{$\tf t \longrightarrow^* \tf t'$}
\DisplayProof
\qquad
\AxiomC{}
\UnaryInfC{$\tf t \longrightarrow^* \tf t'$}
\DisplayProof
\qquad
\AxiomC{$\tf t \longrightarrow^* \tf t'$}
\AxiomC{$\tf t' \longrightarrow^* \tf t''$}
\BinaryInfC{$\tf t \longrightarrow^* \tf t''$}
\DisplayProof
\end{equation*}

\item[3.5.13] To do.

\item[3.5.14] To do.

\item[3.5.16] To do.

\item[3.5.17] To do.

\item[3.5.18] To do.

\item[4.2.1] Because it prevents tail-call optimization, and
fills the stack with useless exception handlers; useless, because
only the innermost one will ever be utilized.

\item[4.2.2] To do.
\end{itemize}
\end{document}
